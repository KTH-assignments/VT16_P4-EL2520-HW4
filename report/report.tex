
\documentclass[a4paper,12pt,oneside,onecolumn]{article} % Document type

\usepackage[left=1.0in, right=1.0in, top=1.0in, bottom=1.0in]{geometry}

\usepackage{fontspec}
\setmainfont{cmr17}
\usepackage{amsmath,amssymb}   % Contains mathematical symbols
\usepackage{xltxtra}
\defaultfontfeatures{Ligatures=TeX}
\usepackage[english]{babel}    % Language
\usepackage[square,numbers]{natbib}     %Nice numbered citations
\bibliographystyle{plainnat}            %Sorted bibliography
\usepackage{float}
\usepackage{pgfplots}
\usepackage{tikz}
\usepackage{lscape}
\usepackage{graphicx}
\usepackage{pstricks}
\usepackage{subfigure}
\usepackage{multicol}
\usepackage{caption}
\usepackage{rotating}
\usepackage{lastpage}
\usepackage{slantsc}
\usepackage{lmodern}
\usepackage{fancyhdr}
\pagestyle{fancy}
\lhead{Filotheou}
\rhead{EL2520}
\chead{Decoupling \& robust loop-shaping}
\rfoot{Page \thepage\ of \pageref{LastPage}}
\cfoot{}
\renewcommand{\footrulewidth}{0.4pt}

\title{
	Computer Exercise 4\\
	EL2520 Control Theory and Practice
}
\author{
  Alexandros Filotheou \\
  alefil@kth.se \\
  871108-5590
}

\newcommand{\image}[3][width=1.0\columnwidth]{
	\begin{figure}[h!]
		\centering
	    \includegraphics[#1]{#2}
		\caption{#3}
		\label{fig:#2}
	\end{figure}
}

\newcommand{\imaget}[3][width=1.0\columnwidth]{
	\begin{figure}[H]
		\centering
      \input{#2}
		\caption{#3}
		\label{fig:#2}
	\end{figure}
}

\begin{document}
	\maketitle

	% Minimum phase case
	\section*{Minimum phase case}

	\subsection*{Dynamic decoupling}
	The dynamic decoupling in exercise 3.2.1 is
	\[
		W(s)=
    \begin{bmatrix}
      1                             & -\dfrac{0.01476}{s + 0.0213} \\
      -\dfrac{0.01336}{s + 0.02572}  & 1                           \\
    \end{bmatrix}
	\]

	\imaget{figures/bode_mp.tex}{Bode diagram of $\tilde{G}(s)$ derived in exercise 3.2.1}
	\imaget{figures/response_mp.tex}{Simulink plots from exercise 3.2.4}

  In this case, inputs $u_1, u_2$ should be paired to outputs $y_1$ and $y_2$
  respectively. Figure \ref{fig:figures/bode_mp.tex} shows that input $u_2$ is
  severely attenuated with respect to output $y_1$; exactly the same happens
  with respect to input $u_1$ and output $y_2$. The controller is thus
  successful.

  Figure \ref{fig:figures/response_mp.tex} exhibits the step responses of the
  closed-loop system. It is evident that the output signal $y_2$, seen here with
  red colour, is not in the least influenced by the preceding step input $u_1$.
  Output $y_1$ exhibits the same behaviour with respect to the step input $u_2$.
  The output signals are now decoupled.


	\subsection*{Glover-MacFarlane robust loop-shaping}

  \imaget{figures/response_mp_gmcf.tex}{Simulink plots from exercise 3.3.4}

  Figure \ref{fig:figures/response_mp_gmcf.tex} illustrates the step responses
  of the closed-loop system after shaping the open loop using the
  Glover-McFarlane method. Both this and the nominal design method result in
  successful decoupling, however the Glover-McFarlane method takes provision
  towards robustness. In this case, comparing the latter method to the former,
  we can see that it results in lower overshoot, but with the usual (although
  small) increase in rise and settling times.

  \newpage

  %-----------------------------------------------------------------------------
	% Non-minimum phase case
	\section*{Non-minimum phase case}

	\subsection*{Dynamic decoupling}
	The dynamic decoupling in exercise 3.2.1 is
	\[
		W(s) =
    \begin{bmatrix}
      \dfrac{0.2}{s+0.2}  & \dfrac{-1.615 s - 0.1386}{s + 0.2}  \\\\
      \dfrac{-1.143 s - 0.1039}{s+0.2}  & \dfrac{0.2}{s + 0.2}  \\
    \end{bmatrix}
	\]

	\imaget{figures/bode_nmp.tex}{Bode diagram of $\tilde{G}(s)$ derived in exercise 3.2.1}
	\imaget{figures/response_nmp.tex}{Simulink plots from exercise 3.2.4}

  In this case, inputs $u_1, u_2$ should be paired to outputs $y_2$ and $y_1$
  respectively. Figure \ref{fig:figures/bode_nmp.tex} shows that input $u_2$ is
  severely attenuated with respect to output $y_2$; exactly the same happens with
  respect to input $u_2$ and output $y_1$. The controller is thus successful.

  Figure \ref{fig:figures/response_nmp.tex} exhibits the step responses of the
  closed-loop system. It is evident that the output signal $y_2$, seen here with
  red colour, is not in the least influenced by the preceding step input $u_2$.
  Output $y_2$ exhibits the same behaviour with respect to the step input $u_1$.
  The output signals are now decoupled.


	\subsection*{Glover-MacFarlane robust loop-shaping}

  \imaget{figures/response_nmp_gmcf.tex}{Simulink plots from exercise 3.3.4}

  Figure \ref{fig:figures/response_nmp_gmcf.tex} illustrates the step responses
  of the closed-loop system after shaping the open loop using the Glover-McFarlane
  method. Both this and the nominal design method result in successful decoupling,
  however the Glover-McFarlane method takes provision towards robustness. In
  this case, comparing the latter method to the former, we can see that it
  results in no overshoot at all, but with a somewhat major increase in rise
  and settling times.

\end{document}
